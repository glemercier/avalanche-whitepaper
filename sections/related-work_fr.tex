\label{section:background}
\tronly{
Bitcoin~\cite{nakamoto2008bitcoin} est une cryptomonnaie qui utilise une blockchain basée sur la preuve de travail
(PoW pour "Proof Of Work") et qui tient un livre de comptes des "sorties de transaction non dépensées", ou UTXO
("Unspent Transaction Output") en anglais. Bien que des techniques basées sur la preuve de
travail~\cite{DworkN92, aspnes2005exposing}, et même d'autres cryptomonnaies dont la génération de pièces est basée
sur la preuve de travail~\cite{vishnumurthy2003karma,rivest1997payword}, ont été explorées avant l'apparition de
Bitcoin, ce dernier fut le premier à incorporer la preuve de travail dans son processus de consensus. Contrairement
aux protocoles résistants aux fautes byzantines traditionnels, Bitcoin apporte une garantie de sûreté probabiliste et
considère que la majorité de la puissance de calcul agit de manière honnête plutôt que de contrôler les membres du
réseau, ce qui a permis de faire émerger un protocole sans permission à l'échelle d'internet. Bien qu'il soit sans
permission et résilient face aux attaques, Bitcoin souffre d'un débit de transaction faible (\textasciitilde{}3 tps)
et d'une latence élevée (\textasciitilde{}5.6 heures pour un réseau comportant 20\% de noeuds byzantins et une garantie
de sûreté de $2^{-32}$). De plus la preuve de travail nécéssite une puissance de calcul considérable qui est consommée
dans le seul et unique but de maintenir la sûreté de fonctionnement.

%0 Bitcoin~\cite{nakamoto2008bitcoin} is a cryptocurrency that uses a blockchain based on
%0 proof-of-work (PoW) to maintain a ledger of UTXO transactions.
%0 While techniques based on proof-of-work~\cite{DworkN92, aspnes2005exposing}, and even cryptocurrencies with minting based on proof-of-work~\cite{vishnumurthy2003karma,rivest1997payword}, have been explored before, Bitcoin was the first to incorporate PoW into its consensus process.
%0 Unlike more traditional BFT protocols, Bitcoin has a probabilistic safety guarantee
%0 and assumes honest majority computational power rather than a known
%0 membership, which in turn has enabled an internet-scale permissionless protocol. While permissionless and resilient to adversaries,
%0 Bitcoin suffers from low throughput (\textasciitilde{}3 tps) and
%0 high latency (\textasciitilde{}5.6 hours for a network with 20\% Byzantine presence and $2^{-32}$ security guarantee).  Furthermore, PoW requires a substantial amount of computational power that is consumed only for the purpose of maintaining
%0 safety.

D'innombrables cryptomonnaies utilisent la preuve de travail~\cite{DworkN92, aspnes2005exposing} dans le but de tenir
un livre de comptes distribué. Tout comme Bitcoin, elles souffrent des mêmes limitations lorsqu'il s'agit de monter en
charge. Il existe plusieurs propositions de protocoles essayant de faire meilleur usage des progrès apportés par la
preuve de travail. Bitcoin-NG~\cite{EyalGSR16} et la version sans permission de Thunderella~\cite{PassS18} utilise
un consensus similaire à celui de Nakamoto pour élire un leader qui va dicter les écritures dans le journal répliqué
pour un temps donné assez long dans le but de fournir un meilleur débit de transactions. En outre, Thunderella apporte
une borne optimiste qui, en considérant 3/4 de puissance de calcul et un leader élu honnêtes, permet de confimer les
transactions rapidement. ByzCoin~\cite{Kokoris-KogiasJ16} sélectionne un échantillon réduit de participants de manière
périodique et exécute ensuite un protocole résistant aux fautes byzantines en pratique (PBFT) entre les noeuds
sélectionnés.

%O % PoW family
%O Countless cryptocurrencies use PoW~\cite{DworkN92, aspnes2005exposing} to maintain a distributed ledger.
%O Like Bitcoin, they suffer from inherent scalability bottlenecks.
%O Several proposals for protocols exist that try to
%O better utilize the effort made by PoW.
%O Bitcoin-NG~\cite{EyalGSR16} and the permissionless version of
%O Thunderella~\cite{PassS18} use Nakamoto-like consensus to elect a leader that
%O dictates writing of the replicated log for a relatively long time so as to
%O provide higher throughput. Moreover, Thunderella provides an
%O optimistic bound that, with 3/4 honest computational power and an honest
%O elected leader, allows transactions to be confirmed rapidly.
%O ByzCoin~\cite{Kokoris-KogiasJ16} periodically selects a small set of
%O participants and then runs a PBFT-like protocol within the selected nodes.
%O %It achieves a throughput of 942 tps with about 35 second latency.

Les protocoles basés sur un accord byzantin~\cite{PeaseSL80, LamportSP82} se basent généralement sur un quorum et
nécéssitent une connaissance précise des membres du réseau. PBFT~\cite{castro1999practical, CL02}, l'un de ces
protocoles les plus connus, nécessite un nombre quadratique d'échanges de messages avant de parvenir à un accord.
Le protocole Q/U~\cite{abd2005fault} et la réplication HQ~\cite{cowling2006hq} utilisent une approche basée sur un
quorum afin d'optimiser le protocole pour obtenir des cas de fonctionnement sans contention et parvenir à un consensus
en un seul tour de communication. Malgré tout, même si ces protocoles améliorent les performances, ils se dégradent
fortement lorsqu'ils sont soumis à une forte contention. Zyzzyva~\cite{KotlaADCW09} ajoute au protocole BFT une
exécution spéculative pour tendre vers un cas de fonctionnement sans erreurs. Les travaux passés portant sur les
systèmes résistants aux fautes byzantines avec contrôle d'accès nécessitent typiquement au moins $f+1$ répliques.
CheapBFT~\cite{kapitza2012cheapbft} tire profit de composants matériels certifiés pour construire un protocole
nécessitant exactement $f+1$ répliques.

%O % Byzantine Agreement family
%O Protocols based on Byzantine agreement~\cite{PeaseSL80, LamportSP82} typically make use of
%O quorums and require precise knowledge of membership.
%O PBFT~\cite{castro1999practical, CL02}, a well-known representative, requires a quadratic number of message exchanges in order to
%O reach agreement.
%O %Some variants are able to scale to dozens of nodes~\cite{ClementWADM09, KotlaADCW09}.
%O The Q/U protocol~\cite{abd2005fault} and HQ replication~\cite{cowling2006hq} use a quorum-based approach to optimize for contention-free cases of operation to achieve consensus in only a single round of communication. However, although these protocols improve on performance, they degrade very poorly under contention. Zyzzyva~\cite{KotlaADCW09} couples BFT with speculative execution to improve the failure-free operation case.
%O %Aliph~\cite{guerraoui2010next} introduces a protocol with optimized performance under several, rather than just one, cases of execution. In contrast, Ardvark~\cite{ClementWADM09} sacrifices some performance to tolerate worst-case degradation, providing a more uniform execution profile. This work, in particular, sacrifices failure-free optimizations to provide consistent throughput even at high number of failures.
%O Past work in permissioned BFT systems typically requires at least $3f+1$ replicas. CheapBFT~\cite{kapitza2012cheapbft} leverages trusted hardware components to construct a protocol that uses $f+1$ replicas.

D'autres travaux tentent d'introduire de nouveaux protocoles qui redéfinissent et relâchent les contraintes du modèle
de résistance aux fautes byzantines. La résistance byzantine à grande échelle~\cite{rodrigues2007large} modifie la
résistance byzantine pratique (PBFT) pour autoriser un nombre arbitraire de répliques et un seuil d'échec, en
fournissant une garantie de vitalité probabilitse pour un taux d'échec donné tout en protégeant la sûreté de
fonctionnement avec une probabilité élevée. Dans une autre forme de relâchement des contraintes,
Zeno~\cite{singh2009zeno} introduit un protocole de réplication de machine d'état résistante aux fautes byzantines qui
échange la cohérence pour une meilleure disponibilité. Plus spécifiquement, Zeno garantit une cohérence assurée au final
plutôt qu'une linéarisation de celle-ci, ce qui veut dire que les participants n'ont pas besoin d'être conhérents mais
doivent se mettrent d'accord au final une fois le réseau stabilisé. En fournissant une garantie de cohérence encore
plus faible, appelée "cohérence causale par embranchement-raccord", Depot~\cite{mahajan2011depot} définit un protocole
garantissant la sûreté de fonctionnement pour $2f+1$ répliques.

NOW~\cite{guerraoui2013highly} uttilise des sous-quorums pour diriger des instances de consensus réduites.
L'enseignement de ce papier est que de petits quorums de taille logarithmique peuvent être extraits à partir d'un
échantillon potentiellement large de noeuds du réseau, ce qui permet à des instances de protocole de consensus plus
petites de tourner en parallèle.

Snow White~\cite{cryptoeprint:2016:919} et Ouroboros~\cite{KiayiasRDO17} font partie des premiers protocoles à la
sûreté prouvée basés sur la preuve d'enjeu (PoS - ou "Proof Of Stake" en anglais). Ouroboros utilise un protocole de
bascule de jeton multi-parties sécurisé dans le but d'incorporer un caractère aléatoire dans l'élection du leader.
Son évolution Ouroboros Praos~\cite{DavidGKR18} apporte la sûreté en présence d'attaquants entièrement adaptatifs.
HoneyBadger~\cite{MillerXCSS16} démontre une bonne vitalité dans un réseau comportant des latences hétérogènes.

Tendermint~\cite{buchman2016tendermint, 1807.04938} change de leader à chaque bloc et a démontré ses qualités dans une
configuration à 64 noeuds. Ripple~\cite{schwartz2014ripple} a une faible latence grâce à l'utilisation de sous-réseaux
certifiés collectivement au sein d'un réseau plus large. La société Ripple génère une liste de noeuds vérifiés par
défaut renouvelée peu souvent, qui en fait un réseau essentiellement centralisé.

HotStuff~\cite{hotstuff,hotstuffpodc} améliore les coûts de communication d'une distribution quadratique vers une
distribution linéaire et simplifie drastiquement les caractéristiques du protocole, même si les limitations liées
à l'élection d'un leader persistent. Facebook utilise Hotstuff comme le consensus principal de son projet Libra.

Dans une configuration synchrone, inspirée par Hoststuff, Sync HotStuff~\cite{synchotstuff} atteint un consensus en
un temps $2\Delta$ avec un coût quadratique et contrairement à d'autres protocoles synchrones à étapes vérouillées, il
opère aussi rapidement que le réseau se propage. Stellar~\cite{mazieres2015stellar} se base sur un accord byzantin
fédéré dans lequel des \emph{tranches de quorum} permettent une confiance hétérogène entre des noeuds différents. La
sûreté de fonctionnement est garantie lorsque les transactions peuvent être reliées entre elles de manière transitive
par des tranches de quorum. Algorand~\cite{GiladHMVZ17} utilise une fonction aléatoire vérifiable pour sélectionner un
comité de noeuds qui participent à un protocole de consensus byzantin innovant.

%0 Other work attempts to introduce new protocols under redefinitions and relaxations of the BFT model.
%0 Large-scale BFT~\cite{rodrigues2007large} modifies PBFT to allow for arbitrary choice of number of replicas and failure threshold, providing a probabilistic guarantee of liveness for some failure ratio but protecting safety with high probability.
%0 In another form of relaxation. Zeno~\cite{singh2009zeno} introduces a BFT state machine replication protocol that trades consistency for high availability. More specifically, Zeno guarantees eventual consistency rather than linearizability, meaning that participants can be inconsistent but eventually agree once the network stabilizes. By providing an even weaker consistency guarantee, namely fork-join-causal consistency, Depot~\cite{mahajan2011depot} describes a protocol that guarantees safety under $2f+1$ replicas.
%0
%0 NOW~\cite{guerraoui2013highly} uses sub-quorums to drive smaller instances of consensus. The insight of this paper is that small, logarithmic-sized quorums can be extracted from a potentially large set of nodes in the network, allowing smaller instances of consensus protocols to be run in parallel.

%O Snow White~\cite{cryptoeprint:2016:919} and
%O Ouroboros~\cite{KiayiasRDO17} are some of the earliest provably secure PoS
%O protocols.  Ouroboros uses a secure multiparty coin-flipping protocol to
%O produce randomness for leader election. The follow-up protocol, Ouroboros
%O Praos~\cite{DavidGKR18} provides safety in the presence of fully adaptive
%O adversaries.
%O HoneyBadger~\cite{MillerXCSS16} provides good liveness in a network with heterogeneous latencies. %and achieves over 819 tps with 5 minute latency on 104 nodes.

%O Tendermint~\cite{buchman2016tendermint, 1807.04938} rotates the leader for each block
%O and has been demonstrated with as many as 64 nodes. Ripple~\cite{schwartz2014ripple} has low latency by utilizing collectively-trusted
%O sub-networks in a large network. The Ripple company provides a
%O slow-changing default list of trusted nodes, which renders the system essentially centralized.
%O HotStuff~\cite{hotstuff,hotstuffpodc} improves the communication cost from quadratic to linear and significantly simplifies the protocol specification, although the leader bottleneck still persists. Facebook uses HotStuff as the core consensus for its Libra project.
%O % Ittai paper
%O In the synchronous setting, inspired by HotStuff, Sync HotStuff~\cite{synchotstuff} achieves consensus in $2\Delta$ time with quadratic cost and unlike other lock-steped synchronous protocols, it operates as fast as network propagates.
%O Stellar~\cite{mazieres2015stellar} uses Federated Byzantine Agreement in which \emph{quorum slices}
%O enable heterogeneous trust for different nodes.  Safety is guaranteed when
%O transactions can be transitively connected by trusted quorum slices.
%O Algorand~\cite{GiladHMVZ17} uses a verifiable random function to select a
%O committee of nodes that participate in a novel Byzantine consensus
%O protocol.
%It achieves over 874 tps with 50 second latency on
%an emulated network of 2000 committee nodes (500K users in total) distributed among 20 cities. To prevent Sybil
%attacks, it uses a mechanism like proof-of-stake that assigns weights to participants in committee selection based on the money in their accounts.

Certains protocoles utilisent une structure de graphe orienté acyclique (DAG) en lieu et place d'une chaîne linéaire
pour atteindre un consensus~\cite{SompolinskyZ15,SompolinskyLZ16,SompolinskyZ18,BentovHMN17,baird2016hashgraph}.
Au lieu de choisir la chaîne la plus longue comme dans Bitcoin, GHOST~\cite{SompolinskyZ15} définit des règles de
sélection de la chaîne qui permettent de prendre en considération des transactions non présentes sur la chaîne
principale, ce qui augmente son efficacité. SPECTRE~\cite{SompolinskyLZ16} utilise des transactions sur le DAG pour
voter de manière récursive via la preuve de travail pour atteindre un consensus, suivi par
SPECTRE~\cite{SompolinskyLZ16} qui obtient un ordre linéaire à partir de tous les blocs. Tout comme PHANTOM, Conflux
obtient également un ordre linéaire entre les transactions via la preuve de travail utilisée sur une structure DAG,
avec une meilleure résistance aux attaques à la vitalité~\cite{confluxLLXLC18}.

Avalanche est différent dans son approche par laquelle le résultat du vote est un reçu instantané déterminé par une
requête sans preuve de travail, tandis que les votes dans PHANTOM ou Conflux sont purement déterminés par la preuve de
travail dans la structure de transactions. A la manière de Thunderella, Meshcash~\cite{BentovHMN17} combine un protocole
lent basé sur la preuve de travail et un protocole de consensus rapide qui apporte une fréquence de génération de blocs
élevée quelque soit la latence du réseau, offrant un temps de confirmation rapide. Hashgraph~\cite{baird2016hashgraph}
est un protocole sans leader qui construit un DAG sur la base d'un "bavardage rendu aléatoire". Il nécessite une
connaissance complète des membres du réseau à tout moment, et comme Ben-Or~\cite{ben1983another}, il s'attend à un
nombre de messages exponentiel~\cite{aspnes2003randomized,CachinV17} pour fonctionner.

%O Some protocols use a Directed Acyclic Graph (DAG) structure instead of a linear chain to achieve
%O consensus~\cite{SompolinskyZ15,SompolinskyLZ16,SompolinskyZ18,BentovHMN17,baird2016hashgraph}.
%O Instead of choosing the longest chain as in Bitcoin,
%O GHOST~\cite{SompolinskyZ15} uses a more efficient chain selection rule that
%O allows transactions not on the main chain to be taken into consideration, increasing efficiency.
%O SPECTRE~\cite{SompolinskyLZ16} uses transactions on
%O the DAG to vote recursively with PoW to achieve consensus, followed up by
%O PHANTOM~\cite{SompolinskyZ18} that achieves a linear order among all blocks.
%O Like PHANTOM, Conflux also finalizes a linear order of transactions by PoW
%O in a DAG structure, with better resistance to liveness attack~\cite{confluxLLXLC18}.
%O Avalanche
%O is different in that the voting result is a one-time chit that is determined by
%O a query without PoW, while the votes in PHANTOM or Conflux are purely determined by PoW in transaction structure.
%O Similar to Thunderella, Meshcash~\cite{BentovHMN17} combines a slow PoW-based protocol with a fast consensus protocol that allows a high block rate regardless of network latency, offering fast confirmation time.
%O Hashgraph~\cite{baird2016hashgraph} is a leader-less protocol that builds a DAG via randomized gossip.
%O It requires full membership knowledge at all times, and, similar to the Ben-Or~\cite{ben1983another}, it requires exponential messages~\cite{aspnes2003randomized,CachinV17} in expectation.
}{
% PoW family
Several proposals for protocols exist that try to
better utilize the effort made by PoW.
Bitcoin-NG~\cite{EyalGSR16} and the permissionless version of
Thunderella~\cite{PassS18} use Nakamoto consensus to elect a leader that
dictates writing of the replicated log for a relatively long time so as to
provide higher throughput. Moreover, Thunderella provides an
optimistic bound that, with 3/4 honest computational power and an honest
elected leader, allows transactions to be confirmed rapidly.
ByzCoin~\cite{Kokoris-KogiasJ16} periodically selects a small set of
participants and then runs a PBFT-like protocol within the selected nodes. 
%It achieves a throughput of 942 tps with about 35 second latency.

% Byzantine Agreement family
Protocols based on Byzantine agreement~\cite{PeaseSL80, LamportSP82} typically make use of
quorums and require precise knowledge of membership.
PBFT~\cite{castro1999practical, CL02}, a well-known representative, requires a quadratic number of message exchanges in order to
reach agreement. Some variants are able to scale to dozens of nodes~\cite{ClementWADM09, KotlaADCW09}.
% However, due to the quadratic message complexity, many variants based on PBFT also inherit this scalability issue.
Snow White~\cite{cryptoeprint:2016:919} and
Ouroboros~\cite{KiayiasRDO17} are some of the earliest provably secure PoS
protocols.  Ouroboros uses a secure multiparty coin-flipping protocol to
produce randomness for leader election. The follow-up protocol, Ouroboros
Praos~\cite{DavidGKR18} provides safety in the presence of fully adaptive
adversaries.
HoneyBadger~\cite{MillerXCSS16} provides good liveness in a network with heterogeneous latencies. %and achieves over 819 tps with 5 minute latency on 104 nodes.
Tendermint~\cite{buchman2016tendermint, 1807.04938} rotates the leader for each block
and has been demonstrated with as many as 64 nodes. Ripple~\cite{schwartz2014ripple} has low latency by utilizing collectively-trusted
sub-networks in a large network. The Ripple company provides a
slow-changing default list of trusted nodes, which renders the system essentially centralized.
HotStuff~\cite{hotstuff,hotstuffpodc} improves the communication cost from quadratic to linear and significantly simplifies the protocol specification, although the leader bottleneck still persists. Facebook uses HotStuff as the core consensus~\cite{librabft} for the Libra project.
% Ittai paper
In the synchronous setting, inspired by HotStuff, Sync HotStuff~\cite{synchotstuff} achieves consensus in $2\Delta$ time with quadratic cost and unlike other lock-steped synchronous protocols, it operates as fast as network propagates.
Stellar~\cite{mazieres2015stellar} uses Federated Byzantine Agreement in which \emph{quorum slices}
enable heterogeneous trust for different nodes.  Safety is guaranteed when
transactions can be transitively connected by trusted quorum slices.

Some protocols use a Directed Acyclic Graph (DAG) structure instead of a linear chain to achieve
consensus~\cite{SompolinskyZ15,SompolinskyLZ16,SompolinskyZ18,BentovHMN17,baird2016hashgraph}.
Instead of choosing the longest chain as in Bitcoin,
GHOST~\cite{SompolinskyZ15} uses a more efficient chain selection rule that
allows transactions not on the main chain to be taken into consideration, increasing efficiency.
SPECTRE~\cite{SompolinskyLZ16} uses transactions on
the DAG to vote recursively with PoW to achieve consensus, followed up by
PHANTOM~\cite{SompolinskyZ18} that achieves a linear order among all blocks.
Avalanche
is different in that the voting result is a one-time chit that is determined by
a query without PoW, while the votes in PHANTOM are purely determined by transaction structure.
Similar to Thunderella, Meshcash~\cite{BentovHMN17} combines a slow PoW-based protocol with a fast consensus protocol that allows a high block rate regardless of network latency, offering fast confirmation time.
Hashgraph~\cite{baird2016hashgraph} is a leader-less protocol that builds a DAG via randomized gossip. 
It requires full membership knowledge at all times, and, similar to the Ben-Or~\cite{ben1983another}, it requires exponential messages~\cite{aspnes2003randomized,CachinV17} in expectation.
}

\begin{comment}
Moreover, in Theorem 5.16, the author mentions if
there is no supermajority virtual vote, all honest nodes will randomly choose
their votes for the next round. Then with non-zero probability, they could
choose the same vote.  The rounds are repeated until all honest nodes
eventually reach the same vote by chance.  This means as the total number of
nodes increases in Hashgraph, the probability of reaching the same vote in a
single trial drops exponentially due to the random voting~\cite{CachinV17}.
\end{comment}
%The paper does not discuss how many rounds of voting are needed in expectation,
%nor\ \ have we been able to obtain open-source code to evaluate Hashgraph.

%\tronly{
%Instead of voting blindly and relying on randomness
%for each agreement, Avalanche creates a positive feedback where votes are
%collected in a separate k-query process determined by current preference
%established locally on each node over time.  We have demonstrated that
%Avalanche is scalable to 2000 nodes distributed world-wide.
%
%The central component of Avalanche is a
%repeated k-query, which serves both as a probabilistic
%broadcast and as a basis for Byzantine agreement.
%As an optimization, these repeated queries are linked
%together via an artificial DAG structure that amortizes the repeated cost of
%querying.  Avalanche does not require overlapping quorums and
%achieves probabilistic consensus with
%$O(kN)$ overall message complexity instead of quadratic
%while allowing for a degree of network churn.
%}

% XXX ADD SAM TOUEG https://zoo.cs.yale.edu/classes/cs426/2017/bib/bracha85asynchronous.pdf
