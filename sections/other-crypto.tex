\subsection{Snowball vs. Blockchain}
Informally speaking, Nakamoto consensus achieved by the Blockchain
algorithm, is a process where the whole system synchronizes their replicated
log by creating a large time delay upon each generation of a block. The
Blockchain system is leaderless because there is no notion of elected leader
that could control the content of multiple slots on the log. However, the peer who is lucky enough
to come up with the block containing a proper nonce will take over the
ownership of the slot, and thus becomes a conceptual temporary ``leader''
only for the slot.  This analogy to traditional BFT reveals the
fundamental inefficiency in Nakamoto consensus, where each commonly accepted
block can be viewed as the outcome of a one-time leader election, the creator
of the block loses leadership immediately after establishment of the block.

There have been work on improving the performance by explicitly stating and
extending the leadership to several slots, such as Bitcoin-NG and Thunderella.
But this makes the protocol leader-based, and may suffer from the issues of
having a single leader in an Internet-scale network environment.

On the other hand, Nakamoto consensus was originally used for Bitcoin, a
peer-to-peer electronic cash system. So each block usually contains hundreds or
thousands of transactions that are mutually compatible (no double spending of
money), and there exists another separated logic, UTXO (Unspent Transaction
Output) graph that is made of all transactions to represent the actual state of
the cash system. On a UTXO graph, a transaction has multiple inputs and
outputs. A valid transaction will provide with signature proofs for all the
inputs that correspond and consume some outputs from some transactions prior in
history, and then offer some dangling unspent outputs which could be taken as
inputs in the future by subsequent transactions. The whole UTXO graph is valid
only if there does not exist two or more transactions that take the same input,
which is usually called a ``double spend''. With that being said,
Nakamoto consensus provides with more power than that is required for
maintaining a double-spend-free UTXO graph, because Blockchain ensures all transactions are
\emph{linearly} ordered, while a valid UTXO graph used for ledger does not
require such linear ordering. For example, suppose there are two transactions
A, B that both have only one input taken from one of the two outputs from
transaction C. So using Blockchain, eventually only a specific linear ordering
of A, B and C will be accepted, which might be unnecessary because both C, A, B
and C, B, A are valid orderings as A and B are independent. The only
necessary dependencies are all exactly captured by UTXO graph itself.

This leads to two possible paths of research: to fully utilize the ability
of keeping a linear log as in state machine replication so Nakamoto consensus is
not wasted, or to construct another weaker consensus that only resolves double
spends, not capable of keeping a linear log but just fits the need for an
electronic cash system. One well-known system for the first approach is
Ethereum, a decentralized application platform that makes use of the linear log
to certify executions and interactions of smart contract logic. The second approach motivates us
to Snowball.


